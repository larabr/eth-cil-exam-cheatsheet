\section{Data Clustering \& Mixture Models}
\subsection*{KMeans}
% ZT?
\textbf{Target:} $\min_{\mathbf{U}, \mathbf{Z}} J(\mathbf{U}, \mathbf{Z}) = \|\mathbf{X} - \mathbf{U} \mathbf{Z}\|_F^2$\\
$= \sum_{n=1}^N \sum_{k=1}^K \mathbf{z}_{k,n} \|\mathbf{x}_n - \mathbf{u}_k\|_2^2$\\
1. \textbf{Initiate:} choose $K$ centroids $\mathbf{U} = [\mathbf{u}_1, \ldots, \mathbf{u}_K]$\\
2. \textbf{Cluster Assign:} data points to clusters. $k^\star(\mathbf{x}_n) = \argmin_k \{ \|\mathbf{x}_n - \mathbf{u}_k\|_2 \}$ returns cluster $k^\star$, whose centroid $\mathbf{u}_{k^\star}$ is closest to data point $\mathbf{x}_n$. Set $\mathbf{z}_{k^\star,n} = 1$, and for $ l \neq k^\star~ \mathbf{z}_{l,n}=0$.\\
3. \textbf{Update centroids}: $\mathbf{u}_k = \frac{\sum_{n=1}^N z_{k,n} \mathbf{x}_n}{\sum_{n=1}^N z_{k,n}}$.\\
4. Repeat from step 2, stops if $\|\mathbf{Z} - \mathbf{Z}^\text{new}\|_0 = \|\mathbf{Z} - \mathbf{Z}^\text{new}\|^2_F = 0$.\\
Computational cost: $O(k\cdot n \cdot d)$


\subsection*{Gaussian Mixture Models (GMM)}
Gaussian $p(x)=\frac{1}{\sqrt{2\pi}\sigma}\mathit{exp}(-\frac{(x-\mu)^2}{2\sigma^2})$
Multivariate $p(x;\mu;\Sigma)=\frac{1}{|\Sigma|^{\frac{1}{2}}(2\pi)^{\frac{D}{2}}}\mathit{exp}[-\frac{1}{2}(x-\mu)^T \Sigma^{-1} (x-\mu)]$ \\
For GMM let $\boldsymbol{\theta}_k = (\boldsymbol{\mu}_k, \boldsymbol{\Sigma}_k)$; $p_{\theta_k}(\mathbf{x}) = \mathcal{N}(\mathbf{x} | \boldsymbol{\mu}_k, \Sigma_k)$, $\boldsymbol{\theta} = (\pi, \theta_1,..,\theta_k})\\
\textbf{Mixture Models:} $p_\theta(\mathbf{x}) = \sum_{k=1}^K \pi_k p_{\theta_k}(\mathbf{x})$\\
\textbf{Assignment variable (generative model):} $z_{ij} \in \{0, 1\}$, $\sum_{j=1}^k z_{ij} = 1$\\
$\operatorname{Pr}(z_k = 1) = \pi_k \Leftrightarrow p(\mathbf{z}) = \prod_{k=1}^K \pi_k^{z_k}$\\
\textbf{Complete data distribution:}\\
$p_\theta(\mathbf{x}, \mathbf{z}) = \prod_{k=1}^K \left( \boldsymbol{\pi}_k p_{\theta_k}(\mathbf{x})\right)^{z_k}$\\
\textbf{Posterior Probabilities:} $\\\operatorname{Pr}(z_k = 1 | \mathbf{x}) = \frac{\operatorname{Pr}(z_k = 1) p(\mathbf{x} | z_k = 1)}{\sum_{l=1}^K \operatorname{Pr}(z_l = 1) p(\mathbf{x} | z_l = 1)} = \frac{\boldsymbol{\pi}_k p_{\theta_k}(\mathbf{x})}{\sum_{l=1}^K \boldsymbol{\pi}_l p_{\theta_l}(\mathbf{x})}$\\
post $p(A|B)=\frac{\text{prior }p(A)\times\text{likelihood } p(B|A)}{\text{evidence } p(B)}$\\
\textbf{Likelihood of observed data $\mathbf{X}$:}\\
$p_\theta(\mathbf{X}) = \prod_{n=1}^N p_\theta(\mathbf{x}_n) = \prod_{n=1}^N \left(\sum_{k=1}^K \pi_k p_{\theta_k}(\mathbf{x}_n)\right)$
\textbf{Max. Likelihood Estimation (MLE):}\\
$\argmax_\theta\sum_{n=1}^N \log \left( \sum_{k=1}^K \pi_k p_{\theta_k}(\mathbf{x}_n)\right)\\
\ge \sum_{n=1^N} \sum_{k=1}^K{q_k[\log p_{\theta_k}(\mathbf{x}_n) + \log \pi_k - \log q_k]}$\\
with $\sum_{k=1}^K{q_k} = 1$ by Jensen Inequality.

\textbf{Generative Model}\\
1. sample cluster index $j \sim Categorical(\pi)$\\
2. given $j$, sample data $x \sim \text{Normal}(\mu_j, \Sigma_j)$


\textbf{Expectation-Maximization (EM) for GMM}\\
E-Step:
Pr$[z_{k,n} = 1 | \mathbf{x}_n] = q_{k, n} = \frac{\boldsymbol{\pi}_k^{(t-1)} \mathcal{N}(\mathbf{x}_n | \boldsymbol{\mu}_k^{(t-1)}, \boldsymbol{\Sigma}_k^{(t-1)})}{\sum_{j=1}^K \boldsymbol{\pi}_j^{(t-1)} \mathcal{N}(\mathbf{x}_n | \boldsymbol{\mu}_j^{(t-1)}, \boldsymbol{\Sigma}_j^{(t-1)})}$\\
M-Step: 
$\boldsymbol{\mu}_k^{(t)} := \frac{\sum_{n=1}^N q_{k,n} \mathbf{x}_n}{\sum_{n=1}^N q_{k,n}}$
$, \boldsymbol{\pi}_k^{(t)} := \frac{1}{N} \sum_{n=1}^N q_{k,n}$\\
$\Sigma_k^{(t)} = \frac{\sum_{n=1}^N q_{k, n} (\mathbf{x}_n - \boldsymbol{\mu}_k^{(t)})(\mathbf{x}_n - \boldsymbol{\mu}_k^{(t)})^\top}{\sum_{n=1}^N q_{k,n}}$

\textbf{Discussion K-means vs. EM}\\
hard assignment vs soft. spherical clusters shapes vs covariance matrix. fast vs slow and more iteration. K-means can be used as initialization for EM.\\
K-means as a special case of GMM with covariances $\Sigma_j = \sigma^2 I$. in the limit of $\sigma \rightarrow 0$, recover K-means (hard assignments).

\textbf{Model Order Selection (AIC / BIC for GMM)}\\
Trade-off between data fit (i.e. likelihood $p(\mathbf{X} | \theta)$) and complexity (i.e. \# of free parameters $\kappa(\cdot)$). For choosing $K$:\\
Akaike Information Criterion: $\operatorname{AIC}(\theta | \mathbf{X}) = -\log p_\theta(\mathbf{X}) + \kappa(\theta)$\\
Bayesian Information Criterion: $\operatorname{BIC}(\theta | \mathbf{X}) = -\log p_\theta(\mathbf{X}) + \frac{1}{2} \kappa(\theta) \log N$\\
\# of free params, fixed covariance matrix: $\kappa(\theta) = K \cdot D + (K - 1)$ ($K$: \# clusters, $D$: $\mathsf{dim}\text{(data)}=\mathsf{dim}(\mu_i)$, $K-1$: $\pi$ of \# free clusters), full covariance matrix: $\kappa(\theta) = K(D + \frac{D(D+1)}{2}) + (K - 1)$.\\
Compare AIC/BIC for different $K$ -- the smaller the better. BIC penalizes complexity more.
